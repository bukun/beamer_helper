
% \usepackage[latin1]{inputenc}

\usepackage{ulem}
\usepackage{mathcomp}  %  Input math sign. Such as .C .
\usepackage{tikz}
\usepackage{smartdiagram}
\usetikzlibrary{shapes,arrows}
\usetikzlibrary{snakes}
% \usetikzlibrary{automata,   positioning}

%%%<
% \usepackage{verbatim}
% \usepackage[active,tightpage] % {preview}
%\PreviewEnvironment{tikzpicture}
%\setlength\PreviewBorder{5pt}%


       % Define block styles
\tikzstyle{decision} = [diamond, draw, fill=blue!20, mininum height=1em,
   text width=3.5em, text badly centered, inner sep=0pt]
\tikzstyle{block} = [rectangle, draw, fill=blue!20,
   text width=5em, text centered, rounded corners, minimum height=2em]
\tikzstyle{line} = [draw, -latex']
\tikzstyle{cloud} = [draw, ellipse,fill=red!20, text width=5em, text centered,
   minimum height=2em]

   \usepackage{tabularx}
\usepackage[no-math]{fontspec}

\usepackage{xeCJK}                       % 这里调用 zhfont.sty
% \usepackage{xkeyval}
\usepackage{xunicode,xltxtra,beamerthemesplit}

\makeatletter %将文献引用作为上标出现,增加括号,
\def\@cite#1#2{\textsuperscript{[{#1\if@tempswa , #2\fi}]}}

\usetheme{CambridgeUS}  % CambridgeUS Warsaw PaloAlto Boadilla  Rochester Luebeck  Berkeley
% \usecolortheme{seahorse}
\usecolortheme{dolphin} % sidebartab dolphin
\makeatletter %将文献引用作为上标出现,增加括号,
\def\@cite#1#2{\textsuperscript{[{#1\if@tempswa , #2\fi}]}}

\setmainfont[BoldFont=Times New Roman]{Times New Roman} % 文中出现的英文
%\setCJKmainfont[BoldFont=微软雅黑]{微软雅黑}  % 文中出现的中文
% \setCJKmonofont{Pragmata}   % 设置公式或代码中出现的中文
%\setsansfont[Mapping=tex-text]{微软雅黑}  %英文无衬线字体

\setCJKmainfont[BoldFont=STXihei, ItalicFont=STKaiti]{STSong}
\setCJKsansfont[BoldFont=SimHei]{STXihei}
\setCJKmonofont{STFangsong}
% \setmonofont{Courier New}
% \setmonofont{UbuntuMono}
\setmonofont{YaHei_Monaco}





%\setCJKmainfont{STKaiti}
% \setCJKmonofont{SimSun}   % 设置等宽字体
% \setsansfont[Mapping=tex-text]{华文细黑}  %英文无衬线字体

\XeTeXlinebreaklocale "zh" %我们处理的文件是中文的
\XeTeXlinebreakskip = 0pt plus 1pt minus 0.1pt %中文自动断行


\definecolor{colorKeyword}{RGB}{0,0,150}
\definecolor{colorNdkeyword}{RGB}{0,100,0}
\definecolor{colorComment}{RGB}{100,100,100}
\definecolor{colorIdentifier}{RGB}{0,0,100}  % 标志:主要颜色
\definecolor{colorString}{RGB}{150,0,0}   % 字符串


\usepackage{listings} % 进行语法高亮显示



\lstset %定义程序代码显示版面,可自定义 。
{language=Python,
% frame=tb,
frame=\leftline\shadowbox,
% frame=shadowbox,
% frame=\topline\bottomline, % 边框
%frame=\single,
numbers=left,
numbersep=5pt,
numberstyle=\ttfamily\footnotesize\color{gray},
stepnumber=1,
numbersep=5pt,
numberstyle=\color[RGB]{80,192,192},
columns=fullflexible,
breaklines=false,  % 是否自动断行
% Frame设置
backgroundcolor=\color{black!10!white},
% backgroundcolor=\color{lightgray},
rulecolor= \color{gray},
% fillcolor= color command
% rulesepcolor= color command
% framexleftmargin=0mm,
% framextopmargin=0mm,
frameround=ffff,
% framesep=8pt, % frame与listing距离,内部。
% rulesep=8pt,  % control the space between frame and listing and between double rules.
framerule=.8pt,  % controls the width of the rules. 边框宽度
% basicstyle=\scriptsize{}\ttfamily{}, % print whole listing small
basicstyle=\linespread{1.1}\ttfamily\footnotesize, % 设置行距,字体  % small  footnotesize
basewidth=0.6em,
% baseheight=0.8em,
keywordstyle=\ttfamily\bf\color{colorKeyword},
ndkeywordstyle=\ttfamily\bf\color{colorNdkeyword},
commentstyle=\color{colorComment},
identifierstyle=\ttfamily\color{colorIdentifier}\bfseries,
stringstyle=\color{colorString}\ttfamily,
% breaklines=true,  % 代码行之间是否有线
extendedchars=true,
% escapeinside=``
emph={czj,joyan},
emphstyle=\color{blue}\bfseries,
keywordstyle=\color{blue}\bfseries,
identifierstyle=\color{black}, % nothing happens
escapeinside=`',
% commentstyle=\color{black}, % white comments
% stringstyle=\ttfamily, % typewriter type for strings
showstringspaces=false
}

% \surroundwithmdframed[
%       hidealllines=true,
%         backgroundcolor=lightgray,
%           innerleftmargin=15pt,
%             innertopmargin=0pt,
%               innerbottommargin=0pt]{lstlisting}


% \lstdefinestyle{shownumber} {numbers=left, stepnumber=1, numberstyle=\tiny, numbersep=10pt}
\lstdefinestyle{nonumber} {numbers=none}


% from: https://gist.github.com/zmwangx/21cda8eb0ab4b1b0e5aa

%
% beamerthemeStanford.sty
%
% Created by Zhiming Wang on August 26, 2014.
%
% This theme is inspired by the CambridgeUS theme packaged with the
% Beamer LaTeX package. The major colors are taken from Stanford's
% official palettes: https://identity.stanford.edu/overview/color.
%
% This work is dedicated by the author to the public domain.
%
% THE SOFTWARE IS PROVIDED "AS IS", WITHOUT WARRANTY OF ANY KIND,
% EXPRESS OR IMPLIED, INCLUDING BUT NOT LIMITED TO THE WARRANTIES OF
% MERCHANTABILITY, FITNESS FOR A PARTICULAR PURPOSE AND
% NONINFRINGEMENT. IN NO EVENT SHALL THE AUTHORS OR COPYRIGHT HOLDERS
% BE LIABLE FOR ANY CLAIM, DAMAGES OR OTHER LIABILITY, WHETHER IN AN
% ACTION OF CONTRACT, TORT OR OTHERWISE, ARISING FROM, OUT OF OR IN
% CONNECTION WITH THE SOFTWARE OR THE USE OR OTHER DEALINGS IN THE
% SOFTWARE.
%

% colors: dark red.
% \definecolor{color_primary}{HTML}{8C1515}
% \definecolor{color_secondary}{HTML}{D2C295}
% \definecolor{color_tertiary}{HTML}{DAD7CB}

% Blue
% \definecolor{color_primary}{HTML}{2D6D79}
% \definecolor{color_secondary}{HTML}{9CB5B1}
% \definecolor{color_tertiary}{HTML}{DCD8D5}

% Blue
% \definecolor{color_primary}{HTML}{005f58}
% \definecolor{color_secondary}{HTML}{52b5f3}
% \definecolor{color_tertiary}{HTML}{f5c904}

% Green
%<<<<
\definecolor{color_primary}{HTML}{16272e}
\definecolor{color_secondary}{HTML}{58765a}
\definecolor{color_tertiary}{HTML}{afee9d}
%>>>>

% base, courtesy of beamerthemeCambridgeUS.sty
\mode<presentation>
\useoutertheme{infolines}
\useinnertheme{rounded}
\setbeamerfont{block title}{size={}}
% custom colors
\usecolortheme{beaver}
\setbeamercolor{frametitle}{fg=color_primary,bg=color_secondary!20}
\setbeamercolor{local structure}{fg=black}
\setbeamercolor{palette primary}{fg=color_primary,bg=color_secondary}
\setbeamercolor{palette secondary}{fg=color_primary,bg=color_tertiary}
\setbeamercolor{palette tertiary}{fg=white,bg=color_primary}
\setbeamercolor{titlelike}{fg=color_primary,bg=color_secondary!20}
% kill certain unprofessional elements
\setbeamertemplate{navigation symbols}{}
\setbeamertemplate{bibliography item}{}
\setbeamertemplate{section in toc}%
{\inserttocsectionnumber.~\inserttocsection}
\setbeamertemplate{subsection in toc}%
{\hspace{1em}$\bullet$~\inserttocsubsection\par}
\setbeamertemplate{itemize items}{$\bullet$}

% plain and definition theorem styles in color_secondary
\makeatletter
\def\th@plain{
  \normalfont
  \setbeamercolor{block title example}{bg=color_secondary,fg=white}
  \setbeamercolor{block body example}{bg=color_secondary!20,fg=black}
  \def\inserttheoremblockenv{exampleblock}
}
\theoremstyle{plain}
\newtheorem{conjecture}{Conjecture}
\newtheorem{proposition}{Proposition}

\def\th@definition{
  \normalfont
  \setbeamercolor{block title example}{bg=color_secondary,fg=white}
  \setbeamercolor{block body example}{bg=color_secondary!20,fg=black}
  \def\inserttheoremblockenv{exampleblock}
}
\theoremstyle{definition}

% highlight theorem style in color_primary
\makeatletter
\def\th@highlight{
  \normalfont
  \setbeamercolor{block title example}{bg=color_primary,fg=white}
  \setbeamercolor{block body example}{bg=color_primary!20,fg=black}
  \def\inserttheoremblockenv{exampleblock}
}
\makeatother
\theoremstyle{highlight}
\newtheorem{hconjecture}{Conjecture}
\newtheorem{hcorollary}{Corollary}
\newtheorem{htheorem}{Theorem}
\newtheorem{question}{Question}
\newtheorem{answer}{Answer}

